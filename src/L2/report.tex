\documentclass[a4paper, 14pt]{extarticle}

\usepackage{../generalPreamble}
\usepackage{../reportFormat}


\begin{document}

\begin{titlepage}
    \centering
    {\bfseries
        \uppercase{
            Минобрнауки России \\
            Санкт-Петербургский государственный \\
            Электротехнический университет \\
            \enquote{ЛЭТИ} им. В.И.Ульянова (Ленина)\\
        }
        Кафедра ИБ

        \vspace{\fill}
        \uppercase{Лабораторная работа №2} \\
        по дисциплине \enquote{Криптография и защита информации} \\
        Тема: Изучение классических шифров Substitution, Permutation/Transposition, Vigenere
    }

    \vspace{\fill}
    \begin{tabularx}{0.8\textwidth}{l X c r}
        Студент гр. 6304 & & \underline{\hspace{3cm}} & Корытов П.В.\\
        Преподаватель & & \underline{\hspace{3cm}} & Племянников А.К.
    \end{tabularx}

    \vspace{1cm}
    Санкт-Петербург \\
    \the\year{}
\end{titlepage}

\newpage

\section*{Цель работы}
Исследовать шифры Substitution, Permutation/Transposition, Vigenere и получить практические навыки работы с ними, в том числе и в программном продукте Cryptool 1 и 2.

\section{Шифр моноалфавитной подстановки}
\subsection{Описание шифра}
Параметры --- Key, Offset
\begin{enumerate}
    \item Первый шаг:
    \begin{itemize}
        \item Удаление всех элементов алфавита, которые присутствуют в кодовом слове
        \item Удвоенные элементы кодового слова слова сливаются в один
    \end{itemize}
    \item Второй шаг
    \begin{itemize}
        \item Задается значение смещение первого элемента кодовго слова (Offset)
        \item По значению смещения определяется количество символов алфавита, полученного после удаления его элементов, которые будут предшествовать вставке когодового слова, после которого запись имеющегося алфавита
    \end{itemize}
\end{enumerate}

\subsection{Формулировка задания}
\begin{enumerate}
    \item Найти шифр в CrypTool 1: Encrypt/Decrypt-> Symmetric (Classic).
    \item Зашифровать и расшифровать текст содержащий только фамилию (транслитерация латиницей) вручную и с помощью шифра c выбранным ключом и смещением Offset≠ 0. Убедиться в совпадении результатов.
    \item Выполнить зашифрование и расшифрование с различными паролями и смещениями Offset и разобраться как формируется алфавит шифрограммы.
    \item Выбрать абзац (примерно 600 символов) из файла English.txt (папка CrypTool/reference) и зашифровать его.
    \item Выполнить атаку на шифротекст, используя приложение из Analysis-> Symmetric Encryption (classic)-> Cipher Text Only.
    \item Повторить шифрование и атаку для тестов примерно в 300 и в 150 символов
    \item Изучите ручное расшифрование для текстов менее 300 символов.
    \item Выбрать новый абзац (примерно 600 символов) из файла English.txt (папка CrypTool/reference) и зашифровать его.
    \item Расшифровать этот абзац, используя приложение Analysis-> Tools for Analysis и Analysis-> Symmetric Encryption (classic)-> Manual Analysis.
    \item  Зашифруйте текст из 200 символов, сохраните ключ, и передайте коллеге для расшифровки.
    \item  Самостоятельно изучите атаки, реализованные CrypTool 2, опираясь на Help и ссылки на статьи.
\end{enumerate}

\subsection{Ход работы}
\lipsum[1]

\section{Шифр двойной перестановки (Permutation / Transposition)}
\subsection{Принцип работы шифра}
Текст записывается в матрицу, производится перестановка строк и столбцов

\subsection{Формулировка задания}
\begin{enumerate}
    \item Найти шифр в CrypTool 1: Encrypt/Decrypt-> Symmetric (Classic).
    \item Зашифровать и расшифровать текст, содержащий ФамилиюИмяОтчество (транслитерация латиницей) вручную и с помощью шифра c ключами для перестановки столбцов и строк. Убедиться в совпадении результатов.
    \item Выполнить зашифрование и расшифрование с различными ключами и с различными вариантами перестановки матрицы с текстом по строкам и столбцам. Разобраться с параметрами утилиты.
    \item Зашифровать текст, содержащий ФамилиюИмяОтчество и провести атаку, основанную на знании исходного текста Analysis-> Symmetric Encryption (classic)-> Known Plaintext. 
    \item Зашифровать текст с произвольным сообщением в формате «DEAR messagе THANKS», используя только одинарную перестановку.
    \item Передайте шифровку соседу, для расшифрования при условии, что формы обращения и завершения письма известны.
    \item Самостоятельно изучите атаки, реализованные в CrypTool 2, опираясь на Help и ссылки на статьи.
\end{enumerate}

\subsection{Ход работы}
\lipsum[1]

\section{Шифр Виженера (Vigenere)}
\subsection{Описание шифра}
\begin{itemize}
    \item Заменим буквы алфавита числами соответствующими их порядковым номерам в алфавите $0, 1, \ldots, n-1$.
    \item Представим символы открытого текста $P_i$, ключа $K_i$ и шифротекста $C_i$
    \item Сформируем \dfn{гамму} повторением ключа: $G=(K_1, \ldots, K_M) \ldots (K_1, \ldots, K_m)$
    \item Шифрование символа: $C_i = (P_i + G_i) \bmod n$
    \item Расшифровка символа: $P_i = (C_i - G_i) \bmod n$
\end{itemize}

\subsection{Формулировка задания}
\begin{itemize}
    \item Найти шифр в CrypTool 1: Encrypt/Decrypt-> Symmetric (Classic).
    \item Зашифровать и расшифровать текст, содержащий только фамилию (транслитерация латиницей) вручную и с помощью шифра c выбранным ключом. Убедиться в совпадении результатов.
    \item Произвести атаку на шифротекст, используя приложение Analysis-> Symmetric Encryption (Classic)-> Cipher Text Only->Vigenere.
    \item Повторить атаку для фрагмента текста из файла English.txt (папка CrypTool/reference). Размер текста не менее 1000 символов.
    \item Воспроизведите эту атаку в автоматизированном режиме: 
    \begin{itemize}
        \item Определите размер ключа с помощью приложения Analysis-> Tools for Analysis-> Autocorrelation
        \item Выполните перестановку текста с размером столбца равным размеру ключа приложением Permutation/Transposition
        \item Определите очередную букву ключа приложением Analysis-> Symmetric Encryption (Classic)-> Cipher Text Only->Caesar.
    \end{itemize}
    \item Самостоятельно изучите атаки, реализованные CrypTool 2,
опираясь на Help и ссылки на статьи.
\end{itemize}

\subsection{Ход работы}
\lipsum[1]

\section*{Выводы}

Использованное ПО --- CrypTool 1 / CrypTool 2 в VirtualBox, neovim и \XeLaTeX{} для написания отчета.


\end{document}
