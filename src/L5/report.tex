\documentclass[a4paper, 14pt]{extarticle}

\usepackage{../generalPreamble}
\usepackage{../reportFormat}
\setcounter{MaxMatrixCols}{20}

\newcommand\sbullet[1][.5]{\mathbin{\vcenter{\hbox{\scalebox{#1}{$\bullet$}}}}}

\begin{document}

\begin{titlepage}
    \centering
    {\bfseries
        \uppercase{
            Минобрнауки России \\
            Санкт-Петербургский государственный \\
            Электротехнический университет \\
            \enquote{ЛЭТИ} им. В.И.Ульянова (Ленина)\\
        }
        Кафедра ИБ

        \vspace{\fill}
        \uppercase{Лабораторная работа №5} \\
        по дисциплине \enquote{Криптография и защита информации} \\
        Тема: Изучение шифра AES
    }

    \vspace{\fill}
    \begin{tabularx}{0.8\textwidth}{l X c r}
        Студент гр. 6304 & & \underline{\hspace{3cm}} & Корытов П.В.\\
        Преподаватель & & \underline{\hspace{3cm}} & Племянников А.К.
    \end{tabularx}

    \vspace{1cm}
    Санкт-Петербург \\
    \the\year{}
\end{titlepage}

\newpage

\section*{Цель работы}
Цель работы: исследовать шифр AES, финалистов конкурса AES, атаку предсказанием дополнения и получить практические навыки работы с шифрами и атакой, в том числе и в программном продукте Cryptool 1 и 2.

\section{Исследование преобразований AES}
\subsection{Формулировка задания}
\begin{enumerate}
    \item Изучить преобразования шифра AES с помощью демонстрационного приложения из Cryptool 1: Indiv.Procedures- >Visualization…->AES->Rijndael Animation.
    \item Выполнить вручную преобразования для одного раунда и вычисление раундового ключа при следующих исходных данных:
    \begin{itemize}
        \item Открытый текст --- фамилия\_имя (транслитерация латиницей)
        \item Ключ – номер группы\_отчество
    \end{itemize}
    \item Проверить полученные результаты с помощью приложения-
инспектора: Indiv.Procedures->Visualization…->AES->Rijndael Inspector.
\end{enumerate}

\subsection{Общее описание шифра}%
\label{subsec:theory_start}
Используется не сеть Фейстеля, а Square-like структура.
\begin{figure}[h]
    \centering
    \includegraphics[width=0.7\textwidth]{img/S001.jpg}
    \caption{Структура шифра}%
\end{figure}

\FloatBarrier{}
\begin{itemize}
    \item Операции проводятся над элементами поля Галуа $GF(2^8)$.\\
    Т.е. байту соответстует многочлен $b_7 x^7 + b_6 x^6 + b_5 x^5 + b_4 x^4 + b_3 x^3 + b_2 x^2 + b_1 x + b_0$
    \item Операция умножения выполняется по модулю неприводимого многочлена $ x^8 + x^4 + x^3 + x + 1 $\\
\end{itemize}
Размер блока --- 16 байт, размер ключа --- 16, 24 или 32 байт. Размер матрицы состояний --- $4\times4$ блока

\begin{figure}[h]
    \centering
    \includegraphics[width=0.3\textwidth]{img/S002.jpg}
    \caption{Матрица состояний}
\end{figure}

\begin{figure}[h]
    \centering
    \includegraphics[width=0.8\textwidth]{img/S003.jpg}
    \caption{Операции шифра AES-192}
\end{figure}

\FloatBarrier{}
\subsection{Операции шифра}
Нижеописанные операции выполняются для всех раундов, кроме MixColumns --- она не выполняется для последнего раунда.
\subsubsection{SubBytes}
Производится нелинейная замена байтов с использованием таблицами Rijndael S-box.

\begin{figure}[h]
    \centering
    \includegraphics[width=0.9\textwidth]{img/S004.jpg}
    \caption{S-box}
\end{figure}
Первая цифра в шестнадцатеричной записи ключа --- строка, вторая --- столбец. Например, ``19'' становится ``d4''.

\FloatBarrier{}
\subsubsection{ShiftRows}
Первая строка сдвигается на 1 байт, вторая --- на 2, третья --- на 3.
\begin{figure}[h]
    \centering
    \begin{subfigure}[b]{0.3\textwidth}
    	\includegraphics[width=\textwidth]{img/S006.jpg}
    	\caption{Вход ShiftRows}
    \end{subfigure}%
    \hspace{1cm}
    \begin{subfigure}[b]{0.3\textwidth}
    	\includegraphics[width=\textwidth]{img/S007.jpg}
    	\caption{Выход ShiftRows}
    \end{subfigure}
\end{figure}

\subsubsection{MixColumns}
Каждый столбец представляется как полином третьем степени. Он умножается в $GF(2^8)$ по модулю $x^4 + 1$ на многочлен $3x^3 + x^2 + x + 2$.
\begin{figure}[h]
    \centering
    \includegraphics[width=0.4\textwidth]{img/S008.jpg}
    \caption{MixColumns}
\end{figure}

\FloatBarrier{}
\subsubsection{AddRoundKey}
Сложение каждого столбца с раундовым ключом с помощью xor.

\begin{figure}[h]
    \centering
    \includegraphics[width=0.3\textwidth]{img/S009.jpg}
    \caption{AddRoundKey}
\end{figure}

\FloatBarrier{}
\subsection{Генерация раундовых ключей}%
\label{subsec:theory_end}
\begin{figure}[h]
    \centering
    \includegraphics{img/S010.jpg}
    \caption{Rcon}
\end{figure}

\begin{enumerate}
    \item К столбцу матрицы ключа применяется побайтовый сдвиг на 1, SubBytes, его xor сложение с первым столбцом матрицы ключа и Rcon
    \begin{figure}[h]
        \centering
        \includegraphics[width=0.5\textwidth]{img/S011.jpg}
        \caption{Первый столбец первого раундового ключа}
    \end{figure}
    \item Оставшиеся слова вычисляются xor'ом предыдущего слова со словом 4 позиции назад
    \begin{figure}[h]
        \centering
        \includegraphics[width=0.5\textwidth]{img/S012.jpg}
        \caption{Второй столбец первого раундового ключа}
    \end{figure}
    \FloatBarrier{}
    \item Второй ключ вычисляется аналогичным образом по первому и т.д.
\end{enumerate}

\subsection{Ход работы}
\begin{enumerate}
    \item Изучены преобразования AES, заполнены разделы~\ref{subsec:theory_start}--\ref{subsec:theory_end}
    \item Для одного раунда преобразования выполнены вручную.\\
    Открытый текст --- \texttt{KORYTOV\_PAVELLLL}, ключ --- \texttt{6304\_VALERIEVICH} 
    \begin{enumerate}
        \item Произведено преобразование ключа и текста в шестнадцатеричный формат:
            \begin{itemize}
                \item Текст --- \texttt{4b4f5259544f565f504156454c4c4c4c}
                \item Ключ --- \texttt{363330345f56414c4552494556494348}\\
            \end{itemize}
            Получившая начальные матрицы:
            \begin{equation}
                \text{Input} = \begin{bmatrix}
                    4b  & 4f  & 52  & 59  \\
                    54  & 4f  & 56  & 5f  \\
                    50  & 41  & 56  & 45  \\
                    4c  & 4c  & 4c  & 4c  \\
                    \end{bmatrix}; \text{Key} = \begin{bmatrix}
                    36  & 33  & 30  & 34  \\
                    5f  & 56  & 41  & 4c  \\
                    45  & 52  & 49  & 45  \\
                    56  & 49  & 43  & 48  \\
                \end{bmatrix}
            \end{equation}
        \item \begin{equation}
                A_1 = AddRoundKey(\text{Input}) = \text{Input} \oplus \text{Key} = \begin{bmatrix}
                    7d & 7c & 62 & 6d \\
                    0b & 19 & 17 & 13 \\
                    15 & 13 & 1f & 00 \\
                    1a & 05 & 0f & 04 \\
                \end{bmatrix}
            \end{equation}
        \item \begin{equation}
                B_1 = SubBytes(A_1) = \begin{bmatrix}
                    ff & 10 & aa & 3c \\
                    2b & d4 & f0 & 7d \\
                    59 & 7d & c0 & 63 \\
                    a2 & 6b & 76 & f2 \\
                \end{bmatrix}
            \end{equation}
        \item \begin{equation}
                C_1 = ShiftRows(B_1) = \begin{bmatrix}
                    ff & 10 & aa & 3c \\
                    d4 & f0 & 7d & 2b \\
                    c0 & 63 & 59 & 7d \\
                    f2 & a2 & 6b & 76 \\
                \end{bmatrix}
        \end{equation}
        \item \begin{equation}
            \begin{split}
                D_1 &= MixColumns(C_1) = \\ &= \begin{bmatrix}
                    02 & 03 & 01 & 01 \\
                    01 & 02 & 03 & 01 \\
                    01 & 01 & 02 & 03 \\
                    03 & 01 & 01 & 02
                \end{bmatrix} \bullet \left( \begin{bmatrix}
                    ff \\ d4 \\ c0 \\ f2
                \end{bmatrix} , \begin{bmatrix}
                    10 \\ f0 \\ 63 \\ a2
                \end{bmatrix}, \begin{bmatrix}
                    aa \\ 7d \\ 59 \\ 6b
                \end{bmatrix} , \begin{bmatrix}
                    3c \\ 2b \\ 7d \\ 76
                \end{bmatrix} \right) = \\
                    &= \begin{bmatrix}
                        b0 & ea & fa & 0e \\
                        e5 & ec & d0 & 9b \\
                        bd & db & d8 & 77 \\
                        f1 & fc & 17 & fe
                    \end{bmatrix}
            \end{split}
        \end{equation}
        \item Раундовый ключ:
        \begin{equation}
            K_1 = \begin{bmatrix}
                1e & 2d & 1d & 29 \\
                31 & 67 & 26 & 6a \\
                17 & 45 & 0c & 49 \\
                4e & 07 & 44 & 0c
            \end{bmatrix}
        \end{equation}
        \item \begin{equation}
            \begin{split}
                A_2 &= AddRoundKey(D_1) = \\ &= \begin{bmatrix}
                        b0 & ea & fa & 0e \\
                        e5 & ec & d0 & 9b \\
                        bd & db & d8 & 77 \\
                        f1 & fc & 17 & fe
                    \end{bmatrix} \oplus \begin{bmatrix}
                1e & 2d & 1d & 29 \\
                31 & 67 & 26 & 6a \\
                17 & 45 & 0c & 49 \\
                4e & 07 & 44 & 0c
            \end{bmatrix} = \begin{bmatrix}
                ae & c7 & e7 & 27 \\
                d4 & 8b & f6 & f1 \\
                aa & 9e & d4 & 3e \\
                bf & fb & 53 & f2
            \end{bmatrix}
        \end{split}
        \end{equation}
    \end{enumerate}
    \item Проверена корректность расчётов с помощью Rijndael Inspector
    % TODO скрин
    \item Проведены наблюдения в Rijndael Flow Visualization. Результаты  на рис.~\ref{img:a:1}
    \begin{figure}[h]
        \centering
        \includegraphics[width=\textheight,angle=90]{img/S013.jpg}
        \caption{Потоковая модель AES}%
        \label{img:a:1}
    \end{figure}
    
\end{enumerate}
\FloatBarrier{}
\section{Исследование финалистов конкурса AES (Rijndael, MARS, Serpent, Twofish)}
\subsection{Формулировка задания}
\begin{enumerate}
    \item Выбрать текст на английском языке (не более 120 знаков)
    \item  Создать бинарный файл с этим текстом, зашифровав и расшифровав его шифром AES на 0-м ключе
    \item  С помощью Cryptool 1 зашифровать c ключом отличным от 0 текст с использованием шифров AES, MARS, RC6, Serpent и Twofish
    \item  Приложением из Cryptool 1 вычислить энтропию исходного текста и шифротекстов, полученных в итоге. Зафиксировать результаты измерений в таблице
    \item  Приложением из Cryptool 1 оцените время проведения атаки «грубой силы» всех шифров для одного и того же шифротекста в случаях, когда известно n-2, n-4, n-6,\ldots{} 2 байт секретного ключа. Зафиксировать результаты измерений в таблице.
\end{enumerate}

\subsection{Ход работы}
\begin{enumerate}
    \item Выбран текст на английском языке: \texttt{Late in the planning of Caesar's assassination, there were two different opinions: one led by Brutus to kill only Caesar} 
    \item Создан бинарный файл
    \begin{figure}[h]
        \centering
        \includegraphics[width=0.7\textwidth]{img/S014.jpg}
        \caption{Бинарный файл}%
        \label{img:a:2}
    \end{figure}
    \item Произведено зашифрование файла всеми алгоритмами-финалистами коонкурса AES
    \begin{figure}[h]
        \centering
        \includegraphics[width=\textwidth]{img/S015.jpg}
        \caption{Шифрование текста}%
        \label{img:}
    \end{figure}
    \FloatBarrier{}
    \item Зафиксирована энтропия исходного текста и полученных шифротекстов
    \begin{table}[h]
        \centering
        \begin{tabular}{@{}ll@{}}
            \toprule
            \textbf{Алгоритм} & \textbf{Энтропия} \\ \midrule
            Исходный текст & $4.10$ \\
            Rijndael (AES) & $6.59$ \\
            MARS & $6.50$ \\
            RC6 & $6.45$ \\
            Serpent & $6.75$ \\
            Twofish & $6.49$ \\ \bottomrule
        \end{tabular}
    \end{table}
    \item Измерено время атаки грубой силой на все шифры
    \begin{table}[h]
        \centering
        \resizebox{\textwidth}{!}{%
            \begin{tabular}{@{}llllllll@{}}
                \toprule
                \multirow{2}{*}{\textbf{Алгоритм}} & \multicolumn{7}{c}{\textbf{Известно байт}} \\
                                                   & \textit{2} & \textit{4} & \textit{6} & \textit{8} & \textit{10} & \textit{12} & \textit{14} \\ \midrule
                Rijndael (AES) & $3*10^{20}$ лет & $4*10^{15}$ лет & $7 * 10^{10}$ лет & $1 * 10^6$ лет & $16$ лет & $2$ часа & $ 1 $ сек \\
                MARS & $4*10^{20}$ лет & $6*10^{15}$ лет & $1.11 * 10^{11}$ лет & $1.6 * 10^6$ лет & $24$ года & $3$ часа & $ 1 $ сек \\
                RC6 & $3*10^{20}$ лет & $4.6*10^{15}$ лет & $7 * 10^{10}$ лет & $1.1 * 10^6$ лет & $16$ лет & $2$ часа & $ 1 $ сек \\
                Serpent & $8*10^{20}$ лет & $1.2*10^{16}$ лет & $1.9 * 10^{11}$ лет & $2.9 * 10^6$ лет & $45$ лет & $5$ часов & $ 1 $ сек \\
                Twofish & $4.8*10^{20}$ лет & $7.3*10^{15}$ лет & $1.11 * 10^{11}$ лет & $1.6 * 10^6$ лет & $26$ лет & $3$ часа & $ 1 $ сек \\ \bottomrule
            \end{tabular}%
        }
    \end{table}
    \end{enumerate}
    \FloatBarrier{}
\section{Атака \enquote{грубой силы} на AES}
\subsection{Формулировка задания}
\begin{enumerate}
    \item Найти и запустить шаблон атаки в CrypTool 2: AES Analysis using Entropy (2).
    \item  Выбрать открытый текст (примерно 1000 знаков) и загрузить его в шаблон.
    \item  Провести атаку «грубой силы» когда известно n-2, n-4, n-6 байт секретного ключа, используя в качестве оценочной функции энтропию и задействовав 1 ядро процессора. Зафиксировать затраты времени.
    \item  Выполнить атаку повторно с средним и максимальным количеством процессорных ядер. Зафиксировать затраты времени.
    \item  Сформировать текст с произвольным сообщением в формате «DEAR SIRS message THANKS» и загрузить его в шаблон. 
    \item  Провести атаку «грубой силы» когда известно n-2, n-4, n-6 байт секретного ключа, используя в качестве оценочной функции
\end{enumerate}

\subsection{Ход работы}
\lipsum[1] %TODO

\section{Атака предссказанием дополнения на шифр AES}
\subsection{Формулировка задания}
\begin{enumerate}
    \item Найти и запустить шаблон атаки в CrypTool 2: Padding Oracle Attack on AES.\@
    \item Подготовьтесь к атаке теоретически
    \item  Внедрите во второй блок исходного текста коды символов своего имени.
    \item  Выполните 3 фазы атаки и сохраните итоговые скриншоты окончанию каждой фазы.  по окончанию каждой фазы
    \item Убедитесь, что атака удалась
\end{enumerate}

\subsection{Ход работы}
\lipsum[1] %TODO

\section*{Выводы}
\lipsum[1] %TODO


\end{document}
