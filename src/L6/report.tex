\documentclass[a4paper, 14pt]{extarticle}
\usepackage{../generalPreamble}
\usepackage{../reportFormat}

\begin{document}
\begin{titlepage}
    \centering
    {\bfseries
        \uppercase{
            Минобрнауки России \\
            Санкт-Петербургский государственный \\
            Электротехнический университет \\
            \enquote{ЛЭТИ} им. В.И.Ульянова (Ленина)\\
        }
        Кафедра ИБ

        \vspace{\fill}
        \uppercase{Отчёт} \\
        по лабораторной работе №6 \\
        по дисциплине \enquote{Криптография и защита информации} \\
        Тема: Изучение хэш-функций
    }

    \vspace{\fill}
    \begin{tabularx}{0.8\textwidth}{l X c r}
        Студент гр. 6304 & & \underline{\hspace{3cm}} & Корытов П.В.\\
        Преподаватель & & \underline{\hspace{3cm}} & Племянников А.К.
    \end{tabularx}

    \vspace{1cm}
    Санкт-Петербург \\
    \the\year{}
\end{titlepage}

\section*{Цель работы}
\lipsum[1] %TODO

\section{Исследование лавинного эффекта MD5, SHA-1, SHA-256, SHA-512}
\subsection{Описание алгоритмов}
\subsubsection{MD5}
\lipsum[1] %TODO

\subsubsection{SHA-1}
\lipsum[1] %TODO

\subsection{Формулировка задания}
\lipsum[1] %TODO

\subsection{Ход работы}
\lipsum[1] %TODO

\section{Хэш-функция SHA-3}
\subsection{Описание алгоритма}
\lipsum[1] %TODO

\subsection{Формулировка задания}
\lipsum[1] %TODO

\subsection{Ход работы}
\lipsum[1] %TODO

\section{Контроль целостности по коду HMAC}
\subsection{Описание алгоритма}
\lipsum[1] %TODO

\subsection{Формулировка задания}
\lipsum[1] %TODO

\subsection{Ход работы}
\lipsum[1] %TODO

\section{Атака дополнительной коллизии на хэш-функции}
\subsection{Описание атаки}
\lipsum[1] %TODO

\subsection{Формулировка задания}
\lipsum[1] %TODO

\subsection{Ход работы}
\lipsum[1] %TODO

\section*{Выводы}
\lipsum[1] %TODO

\end{document}
