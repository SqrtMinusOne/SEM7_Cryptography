\documentclass[a4paper, 14pt]{extarticle}
\usepackage{../generalPreamble}
\usepackage{../reportFormat}

\DeclareMathOperator{\floor}{floor}

% chktex-file 29

\begin{document}
\begin{titlepage}
    \centering
    {\bfseries
        \uppercase{
            Минобрнауки России \\
            Санкт-Петербургский государственный \\
            Электротехнический университет \\
            \enquote{ЛЭТИ} им. В.И.Ульянова (Ленина)\\
        }
        Кафедра ИБ

        \vspace{\fill}
        \uppercase{Отчёт} \\
        по лабораторной работе №6 \\
        по дисциплине \enquote{Криптография и защита информации} \\
        Тема: Изучение хэш-функций
    }

    \vspace{\fill}
    \begin{tabularx}{0.8\textwidth}{l X c r}
        Студент гр. 6304 & & \underline{\hspace{3cm}} & Корытов П.В.\\
        Преподаватель & & \underline{\hspace{3cm}} & Племянников А.К.
    \end{tabularx}

    \vspace{1cm}
    Санкт-Петербург \\
    \the\year{}
\end{titlepage}

\section*{Цель работы}
Исследование хэш-функций MD5, SHA-256, SHA-512, SHA-3, кода контроля целостности HMAC и анализ атак дополнительной коллизии на хэш-функцию. Получить практические навыки работы с хэш- функциями и атакой на них, в том числе и в программном продукте Cryptool 1 и 2.

\section{Исследование лавинного эффекта MD5, SHA-1, SHA-256, SHA-512}
\subsection{Описание алгоритмов}
\subsubsection{MD5}
MD5 перерабатывает сообщение произвольной длины в сообщение длины $128$ бит.

Сообщение дополняется, чтобы длина делилась на $512$
\begin{itemize}
    \item Добавляется бит 1
    \item Дописываются нули, чтобы осталось 64 бита
    \item В последние 64 бита записывается длина сообщения по модулю $2^{64}$\\
\end{itemize}

После чего сообщение делится на блоки по $512$ бит. Для каждого блока выполняет 4 раунда по 16 операций.

Структура одной операции MD5 представлена на рис.~\ref{img:md5}

\begin{figure}[h]
    \centering
    \includegraphics[width=0.5\textwidth]{img/S001.jpg}
    \caption{Одна операция MD5}%
    \label{img:md5}
\end{figure}

\FloatBarrier{}

MD5 оперирует на $128$-битном состоянии, поделённом на 4 слова по $32$ бит. 
\begin{itemize}
    \item Инициализация для 1-го блока:
    \begin{itemize}
        \item $A := 0x67452301$
        \item $B := 0xefcdab89$
        \item $ C:= 0x98badcfe $
        \item $D := 0x10325476$
    \end{itemize}
    \item $F$ --- нелинейная функция:
    \begin{equation}
        F(B,C,D) = \begin{cases}
            (B \wedge C) \vee (\neg B \wedge D), &\text{ если } i \in [0, 15]\\
            (B \wedge D) \vee (C \wedge \neg D), &\text{ если } i \in [16, 31]\\
            B \oplus C \oplus D, &\text{ если } i \in [32, 47]\\
            C \oplus (B \vee \neg D) &\text{ если } i \in [48, 63]\\
        \end{cases}
    \end{equation}
    \item $M_i$ --- часть входного блока размером $32$ бит. Выбор номера блока:
    \begin{equation}
        \begin{cases}
            i, &\text{ если } i \in [0, 15]\\
            (5i + 1) \bmod 16, &\text{ если } i \in [16, 31]\\
            (3i + 5) \bmod 16 &\text{ если } i \in [32, 47]\\
            7i \bmod 16 &\text{ если } i \\
        \end{cases}
    \end{equation}
    \item $K_i$ --- константа такого же размера, разная для каждой операции
    \begin{equation}
        K_i = \floor (2^{32} \bullet |\sin(i+1)|)
    \end{equation}
    \item ${<<<}_s$ --- сдвиг влево на $s$ бит
    \begin{itemize}
        \item $s[0, 15] = \{7, 12, 17, 22,  7, 12, 17, 22,  7, 12, 17, 22,  7, 12, 17, 22\}$
        \item $s[16, 31] = \{ 5,  9, 14, 20,  5,  9, 14, 20,  5,  9, 14, 20,  5,  9, 14, 20 \}$
        \item $s[32, 47] = \{ 4, 11, 16, 23,  4, 11, 16, 23,  4, 11, 16, 23,  4, 11, 16, 23 \} $
        \item $s[48, 63] = \{ 6, 10, 15, 21,  6, 10, 15, 21,  6, 10, 15, 21,  6, 10, 15, 21 \}$
    \end{itemize}
    \item $\boxplus$ --- сложение по модулю $2^{32}$\\
\end{itemize}

\subsubsection{SHA-1}
Длина хэша SHA-1 --- $160$ бит. Размер блока --- $512$ бит.

Размер блоков и правила дополнения совпадают с таковыми для MD5. Вид одной операции представлен на рис.~\ref{img:sha1}
\begin{figure}[h]
    \centering
    \includegraphics[width=0.5\textwidth]{img/S002.jpg}
    \caption{Одна итерация SHA-1}%
    \label{img:sha1}
\end{figure}

\FloatBarrier{}

\begin{itemize}
    \item Инициализация:
        \begin{itemize}
            \item $h0 = \mathtt{0x67452301}$
            \item $h1 = \mathtt{0xefcdab89}$
            \item $h2 = \mathtt{0x98badcfe}$
            \item $h3 = \mathtt{0x10325476}$
            \item $h4 = \mathtt{0xc3d2e1f0}$
    \end{itemize}
    \item $W_i$ --- расширение $16$ $32$-х слов входного блока до $80$. Для $i > 15$:
    \begin{equation}
        W_i = (W_{i-3} \oplus W_{i-8} \oplus W_{i-14} \oplus W_{i-16}) \lll 1
    \end{equation}
    \item $F$ такая же, как в MD5, но с промежутками по $20$ вместо $16$
    \item $K_i$ --- константа:
    \begin{equation}
        K = \begin{cases}
            \mathtt{0x5A827999} , &\text{ если } i \in [0, 19]\\
            \mathtt{0x6ED9EBA1} , &\text{ если } i \in [20, 39]\\
            \mathtt{0x8F1BBCDC} , &\text{ если } i \in [40, 59]\\
            \mathtt{0xCA62C1D6} , &\text{ если } i \in [60, 79]\\
        \end{cases}
    \end{equation}
\end{itemize}

\subsection{Формулировка задания}
\begin{itemize}
    \item  Открыть текст не менее 1000 знаков. Добавить свое ФИО последней строкой. Перейти к утилите Indiv.Procedures->Hash->Hash Demonstration..
    \item  Задать хэш-функцию, подлежащую исследованию: MD5, SHA-1, SHA-256, SHA-512.
    \item  Для каждой хэш-функции повторить следующие действия:
    \begin{itemize}
        \item  Измените (добавлением, заменой, удалением символа) исходный файл
        \item  Зафиксировать количество измененных битов в дайджесте модифицированного сообщения.
        \item  Вернуть сообщение в исходное состояние.
    \end{itemize}
    \item  Выполните процедуру 3 раза (добавлением, заменой, удалением символа) и подсчитайте среднее количество измененных бит дайджеста.  Зафиксировать результаты в таблице.
\end{itemize}

\subsection{Ход работы}
\begin{enumerate}
    \item Выбран и модифицирован исходный текст длиной 1686 знаков.
        \begin{figure}[h]
            \centering
            \includegraphics[width=0.7\textwidth]{img/S004.jpg}
            \caption{Исходный текст}%
            \label{img:s1:1}
        \end{figure}

    \item С помощью указанной утилиты произведено исследование изменения дайджеста для указанных хэш-функций

        \begin{table}[h]
            \centering
            \begin{tabular}{@{}lllll@{}}
                \toprule
                    & \textbf{MD5} & \textbf{SHA-1} & \textbf{SHA-256} & \textbf{SHA-512} \\ \midrule
                \textit{Добавление} & 53.91        & 50.61          & 51.95            & 49.22            \\
                \textit{Удаление}   & 53.13        & 46.25          & 54.30            & 47.46            \\
                \textit{Изменение}  & 55.47        & 43.13          & 48.83            & 52.93            \\
                \textit{Среднее}    & 54.17        & 46.63          & 51.69            & 49.87            \\ \bottomrule
            \end{tabular}
        \end{table}
        
        \begin{figure}[h]
            \centering
            \includegraphics[width=0.7\textwidth]{img/S005.jpg}
            \caption{Hash Demonstration}%
            \label{img:}
        \end{figure}
\end{enumerate}

\section{Хэш-функция SHA-3}
\subsection{Описание алгоритма}
SHA-3 построен на основе криптографической губки. Вид этой структуры представлен на рис.~\ref{img:sponge}

\begin{figure}[h]
    \centering
    \includegraphics[width=0.8\textwidth]{img/S003.jpg}
    \caption{Губка}%
    \label{img:sponge}
\end{figure}

\FloatBarrier{}
\subsubsection{Дополнение}
Сообщение должно быть поделено на блоки по $r$ бит. Для этого к сообщению добавляется блок вида $10\ldots01$.

Если последний блок имеет длину $r-1$, то он дополняется единицей, следующий блок состоит из $r-1$ нулей и $1$.

Если длина последнего блока $r$, то все равно добавляется блок описанного вида.

\subsubsection{Устройство губки}
На входе:
\begin{itemize}
    \item $P$ разбивается на блоки $P_0, \ldots, P_{n-1}$ по $r$ бит
    \item Инциализация $S$ нулевым вектором
    \item Впитывание (Absorbing):
        \begin{itemize}
            \item $P_i$ дополняется $c$ нулями до длины блока $b$
            \item Это побитово складывается с $S$
            \item Состояние модифицируется функцией $f$
        \end{itemize}
    \item Отжатие (Squeezing)\\
    На каждом шаге от $S$ сохраняются первые $r$ байт, после чего применяется $f$. Так повторяется, пока не получится выход $Z$ новой длины

\end{itemize}

\subsection{Формулировка задания}
\begin{itemize}
    \item  Открыть шаблон Keccak Hash (SHA-3) в Cryptool 2
    \item  В модуле Keccak сделать следующие настройки:
    \begin{itemize}
        \item  Adjust manually=ON
        \item  Keccak version= SHA3--512
    \end{itemize}
    \item  Загрузить файл из предыдущего задания
    \item  Запустить проигрывание шаблона в режиме ручного управления:
    \begin{itemize}
        \item  Сохранить скриншоты преобразований первого раунда
        \item  Сохранить скриншот заключительной фазы
        \item  Сохранить значение дайджеста
    \end{itemize}
    \item  Вычислить значения дайджеста для модифицированных текстов из предыдущего задания
    \item  Подсчитать лавинный эффект с помощью самостоятельно разработанной автоматизированной процедуры
\end{itemize}

\subsection{Ход работы}
\begin{enumerate}
    \item Открыт и настроен указанный шаблон с указанным файл
    \item Проведены преобразования первого раунда
        \begin{figure}[h]
            \centering
            \includegraphics[width=\textwidth]{img/S006.jpg}
            \caption{Первое подмешивание}%
            \label{img:}
        \end{figure}

        \begin{figure}[h]
            \centering
            \includegraphics[width=\textwidth]{img/S007.jpg}
            \caption{Структура $f$}%
        \end{figure}

        \begin{figure}[h]
            \centering
            \includegraphics[width=\textwidth]{img/S008.jpg}
            \caption{$\theta$}%
        \end{figure}

        \begin{figure}[h]
            \centering
            \includegraphics[width=\textwidth]{img/S009.jpg}
            \caption{$\rho$}%
        \end{figure}

        \begin{figure}[h]
            \centering
            \includegraphics[width=\textwidth]{img/S010.jpg}
            \caption{$\pi$}%
        \end{figure}

        \begin{figure}[h]
            \centering
            \includegraphics[width=\textwidth]{img/S011.jpg}
            \caption{$\chi$}%
        \end{figure}

        \begin{figure}[h]
            \centering
            \includegraphics[width=\textwidth]{img/S012.jpg}
            \caption{$\iota$}%
        \end{figure}

        \begin{figure}[h]
            \centering
            \includegraphics[width=\textwidth]{img/S013.jpg}
            \caption{Результаты отжатия}%
        \end{figure}

        \FloatBarrier{}
    \item Создана автоматизированная процедура для оценки лавинного эффекта. Она представлена на рис.~\ref{img:s2:proc}
        \begin{figure}[h]
            \centering
            \includegraphics[width=\textwidth]{img/S014.jpg}
            \caption{Процедура для оценки лавинного эффекта}%
            \label{img:s2:proc}
        \end{figure}

        Результаты:
        \begin{table}[h]
            \centering
            \begin{tabular}{@{}ll@{}}
                \toprule
                    & \textbf{SHA-3} \\ \midrule
                \textit{Добавление} & 99.8           \\
                \textit{Удаление}   & 53.13          \\
                \textit{Изменение}  & 100            \\
                \textit{Среднее}    & 99.8           \\ \bottomrule
            \end{tabular}
        \end{table}

\end{enumerate}

\section{Контроль целостности по коду HMAC}
\subsection{Описание механизма}
HMAC позволяет контролировать целостность данных, передаваемых в ненадежной среде. Два клиента разделяют общий секретный ключ.
\begin{equation}
    HMAC_k(\mathtt{text}) = H((K \oplus \mathtt{opad}) || H ((K \oplus \mathtt{ipad}) || \mathtt{text}) ),
\end{equation}
где:
\begin{itemize}
    \item $||$ --- конкатенация;
    \item $K$ --- секретный ключ;
    \item \texttt{ipad} --- блок, где байт \texttt{0x36} повторяется $b$ раз;
    \item \texttt{opad} --- блок, где байт \texttt{0x5c} повторяется $b$ раз;
    \item $H$ --- хэш-функция.
\end{itemize}

\subsection{Формулировка задания}
\begin{enumerate}
    \item  Выбрать текст на английском языке (не менее 1000 знаков), добавить собственное ФИО и сохранить в файле формата TXT
    \item  Придумать пароль и сгенерировать секретный ключ утилитой Indiv.Procedures->Hash-> Key Generation из Cryptool 1. Сохранить ключ в файле формата TXT.\@ Прочитать Help к этой утилите.
    \item  Сгенерировать HMAC для имеющегося текста и ключа с помощью утилиты Indiv.Procedures->Hash-> Generation of HMACs. Сохранить HMAC в файле формата TXT.\@ Прочитать Help к этой утилите.
    \item  Передать пароль, HMAC (и его характеристики), исходный текст и модифицированный текст коллеге, не раскрывая, какой текст является корректным. Попросите коллегу определить это самостоятельно.
\end{enumerate}

\subsection{Ход работы}
\begin{enumerate}
    \item Использован тот же самый текст
    \item Сгенерирован секретный ключ. Результаты на рис.~\ref{img:s3:1}
    \begin{figure}[h]
        \centering
        \includegraphics[width=0.7\textwidth]{img/S015.jpg}
        \caption{Генерация секретного ключа}%
        \label{img:s3:1}
    \end{figure}
    
    \FloatBarrier{}
    \item Сгенерирован HMAC для сообщения
    \begin{figure}[h]
        \centering
        \includegraphics[width=\textwidth]{img/HMAC.jpg}
        \caption{Генерация HMAC}%
    \end{figure}
    
\end{enumerate}

\section{Атака дополнительной коллизии на хэш-функции}
\subsection{Описание атаки}
Основана на парадоксе дней рождения.

Пусть $P(n)$ --- вероятность, что в группе из $n$ человек хотя бы двое имеют одинаковый день рождения. Количество дней в году --- $N$.
\begin{equation}
    \begin{split}
        P(N) &= 1 - 1 \bullet \left( 1 - \frac{1}{N} \right) \bullet \left( 1 - \frac{2}{N} \right) \bullet \ldots \bullet \left( 1 - \frac{n-1}{N} \right) = \\
             &= 1 - \prod_{i=0}^{n-1} \left( 1 - \frac{i}{N} \right)
    \end{split}
\end{equation}

Поскольку при $x \ll |x|$: $e^x \approx 1 + x$,
\begin{equation}
    P(n) \approx 1 - \prod_{i=0}^{n-1} \exp{(-i/N)} = 1 - \exp \sum_{i=0}^{n-1} (-i / N) = 1 - e^{-n(n-1)/2N}
\end{equation}

Попробуем найти такое $n$, чтобы $P(n) \ge 0.5$:
\begin{equation}
    \frac{1}{2} \ge e^{- \dfrac{n(n-1)}{2N} } \ge e^{- \dfrac{n^2}{2N}} \Rightarrow n \ge \sqrt{2 \ln 2 \bullet N}
\end{equation}

Для $N=365$ такое $n$ --- 23.

Пусть $h$ --- хэш-функция. Нужно найти $x_1 \ne x_2$, такие, что $h(x_1) = h(x_2)$. Длина хэш-функции --- $L$.

Сложность такой атаки оценивается как $o(2^{L/2})$

\subsection{Формулировка задания}
\begin{itemize}
    \item  Сформировать два текста на английском языке --- один истинный, а другой фальсифицированный. Сохранить тексты в файлах формата *.txt
    \item  Утилитой Analysis-> Attack on the hash value\ldots{} произвести модификацию сообщений для получения одинакового дайджеста. В качестве метода модификации выбрать Attach characters-> Printable characters.
    \item  Проверить, что дайджесты сообщений действительно совпадают с заданной точностью.
    \item  Сохранить исходные тексты, итоговые тексты и статистику атаки для отчета.
5. Зафиксировать временную сложность атаки для 8, 16, 32,40, 48, … бит совпадающих частей дайджестов.
\end{itemize}

\subsection{Ход работы}
\begin{enumerate}
    \item Сформированы два текста на английском языке. \\
        Исходный: \texttt{Russia is still trying to expand its sphere of influence and acquire more buffer states. With the exception of the eastern frontier, where nothing can be done because of China, it has steadily supported a number of allied states like Kazakhstan or managed to force the local power to concede its foreign policy.}

        Модифицированный: \texttt{Russia is not trying to expand its sphere of influence and acquire more buffer states. With the exception of the western frontier, where nothing can be done because of USA, it has steadily helped a number of peaceful states like Belarus or managed to persuade the local power to follow its foreign policy.} 
    \item Проведена атака на значение хэша для получения совпадения первых 16 бит
        \begin{figure}[h]
            \centering
            \includegraphics[width=0.7\textwidth]{img/S018.jpg}
            \caption{Статистика атаки}
        \end{figure}

        \begin{figure}[h]
            \centering
            \includegraphics[width=\textwidth]{img/S019.jpg}
            \caption{Модифицированные тексты}%
            \label{img:}
        \end{figure}

        Значения первых двух байт функции MD5 действительно совпадают для обоих текстов.
    \item Оценено время атаки для разного количества совпадающих бит дайджеста

        \begin{table}[h]
            \centering
            \begin{tabular}{@{}ll@{}}
                \toprule
                \textbf{Совпадение дайджеста (бит)} & \textbf{Время атаки}   \\ \midrule
                \textit{8}                          & $0$ с                  \\
                \textit{16}                         & $0$ с                  \\
                \textit{24}                         & $0$ с                  \\
                \textit{32}                         & $1$ с                  \\
                \textit{40}                         & $7$ с                  \\
                \textit{48}                         & $2$ мин                \\
                \textit{56}                         & $24$ мин               \\
                \textit{64}                         & $6$ часов              \\
                \textit{72}                         & $4.6$ дня              \\
                \textit{80}                         & $71.2$ дня             \\
                \textit{88}                         & $3.2$ года             \\
                \textit{96}                         & $52$ года              \\
                \textit{104}                        & $8.2 \bullet 10^2$ лет \\
                \textit{112}                        & $1.3 \bullet 10^4$ лет \\
                \textit{120}                        & $2.1 \bullet 10^4$ лет \\
                \textit{128}                        & $7 \bullet 10^{93}$ лет  \\ \bottomrule
            \end{tabular}
        \end{table}
        \FloatBarrier{}
\end{enumerate}

\section*{Выводы}
Исследовано применение хэш-функций MD5, SHA-1, SHA-256, SHA-512, SHA-3, контроль целостности по коду HMAC и атака дополнительной коллизии.

\begin{table}[h]
    \centering
    \begin{tabular}{@{}llllll@{}}
        \toprule
        \textbf{Хэш-функция}            & \textbf{MD5} & \textbf{SHA-1} & \textbf{SHA-256} & \textbf{SHA-512} & \textbf{SHA-3-512} \\ \midrule % chktex 8
        \textit{Длина дайджеста}        & 128          & 160            & 256              & 512              & 512                \\
        \textit{Размер блока обработки} & 128          & 160            & 256              & 512              & 1600               \\
        \textit{Размер блока}           & 512          & 512            & 512              & 1024             & 576                \\
        \textit{Число итераций}         & 64           & 80             & 64               & 80               & 24                 \\ \bottomrule
    \end{tabular}
\end{table}

Лавинный эффект функций MD5, SHA-1, SHA-256, SHA-512 выражается в том, что при изменении одного символа изменяется примерно $50\%$ дайджеста.

Для функции SHA-3 лавинный эффект выражается в почти полном изменении дайджеста. Для размера выхода до $512$ бит стойкость SHA-3 неотличима от идеальной хэш-функции.

Механизм HMAC может быть использован для контроля целостности данных при передаче по ненадежному каналу, но для этого клиентам нужен общий секретный ключ, который нужно передать по закрытому каналу.

Для хэш-функции с n-битным значением сложность атаки дополнительной коллизии --- поиска двух разных значений с одинаковыми хэш-кодами --- примерно равна $2^{n/2}$

\end{document}
