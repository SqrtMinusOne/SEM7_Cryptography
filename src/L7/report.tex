\documentclass[a4paper, 14pt]{extarticle}
\usepackage{../generalPreamble}
\usepackage{../reportFormat}

\begin{document}
\begin{titlepage}
    \centering
    {\bfseries
        \uppercase{
            Минобрнауки России \\
            Санкт-Петербургский государственный \\
            Электротехнический университет \\
            \enquote{ЛЭТИ} им. В.И.Ульянова (Ленина)\\
        }
        Кафедра ИБ

        \vspace{\fill}
        \uppercase{Отчёт} \\
        по лабораторной работе №7 \\
        по дисциплине \enquote{Криптография и защита информации} \\
        Тема: Изучение ассиметричных шифров
    }

    \vspace{\fill}
    \begin{tabularx}{0.8\textwidth}{l X c r}
        Студент гр. 6304 & & \underline{\hspace{3cm}} & Корытов П.В.\\
        Преподаватель & & \underline{\hspace{3cm}} & Племянников А.К.
    \end{tabularx}

    \vspace{1cm}
    Санкт-Петербург \\
    \the\year{}
\end{titlepage}
\section*{Цель работы}
Исследовать протокол Диффи-Хеллмана, шифр RSA и получить практические навыки работы с ними, в том числе и в программном продукте CrypTool 1.

\section{Протокол Диффи-Хеллмана}
\subsection{Описание протокола}
\lipsum[1] %TODO

\subsection{Формулировка задания}
\begin{enumerate}
    \item Запустите утилиту Indiv.Procedures->Protocols->Diffie-Hellman demonstration… и установите все опции информирования в ON.\@
    \item  Выполните последовательно все шаги протокола.
    \item  Сохраните лог-файл протокола для отчета (пиктограмма с изображением ключа).
    \item  Используйте полученный общий ключ для зашифровки и расшифровки произвольного сообщения. Шифр выберите самостоятельно.
\end{enumerate}

\subsection{Выполнение задания}
\lipsum[1] %TODO

\section{Шифр RSA}
\subsection{Описание шифра}
\lipsum[1] %TODO

\subsection{Формулировка задания}
\begin{itemize}
    \item 1. Запустите Demonstration утилиту Indiv.Procedures->RSACryptisystem->RSA
    \item  Задайте в качестве обрабатываемого сообщения свою Ф.И.О.
    \item  Сгенерируйте открытый и закрытый ключи.
    \item  Зашифруйте сообщение. Сохраните скриншот результата.
    \item  Расшифруйте сообщение. Сохраните скриншот результата.
    \item Убедитесь, что расшифрование произошло корректно.
\end{itemize}

\subsection{Выполнение задания}
\lipsum[1] %TODO

\section{Исследование шифра RSA}
\subsection{Формулировка задания}
\begin{itemize}
    \item 1. Выбрать текст на английском языке (не менее 1000 знаков) и сохранить в файле формата *.txt
    \item  Сгенерировать пары ассиметричных RSA-ключей утилитой Digital Signatures->PKI->Generate/Import Keys с различными длинами (4 варианта).
    \item  Зашифровать текст (примерно 1000 символов) различными открытыми ключами. Зафиксировать время зашифровки.
    \item  Расшифровать текст различными закрытыми ключами. Зафиксировать время зашифровки.
    \item  Проверить корректность расшифровки. Зафиксировать скриншоты результата.
\end{itemize}

\subsection{Выполнение задания}
\lipsum[1] %TODO

\section{Атака грубой силы на RSA}
\subsection{Формулировка задания}
\begin{itemize}
    \item 1. Запустите утилиту Indiv.Procedures->RSACryptosystem->RSA Demonstration
    \item  Установите переключатель в режим «Choose two prime…».
    \item  Выберите параметры $p$ и $q$ так, чтобы $n=pq> 256$.
    \item  Задайте открытый ключ $e$.
    \item  Зашифруйте произвольное сообщение и передайте его вместе $с$, $n$ и $e$ коллеге. В ответ получите аналогичные данные от коллеги.
    \item  Запустите утилиту Indiv.Procedures->RSACryptosystem->RSA Demonstration и установите переключатель в режим «For data encryption…»
    \item  Выполните факторизацию модуля n командой Factorize…
    \item  Используйте полученный результат для расшифровки сообщения полученного от коллеги. Проверьте корректность.
\end{itemize}

\subsection{Выполнение задания}
\lipsum[1] %TODO

\section{Имитация атаки на гибридную криптосистему}
\subsection{Описание}
\lipsum[1] %TODO

\subsection{Формулировка задания}
\begin{itemize}
    \item 1. Подготовьте текст передаваемого сообщения на английском с вашим именем в конце.
    \item  Запустите утилиту Analysis->Asymmetric Encr…->Side-Channel attack on «Textbook RSA»…
    \item  Настройте сервер, указав в качестве ключевого слова ваше имя, используемое в конце текста.
    \item  Выполните последовательно все шаги протокола.
    \item  Сохраните лог-файлы участников протокола для отчета.
\end{itemize}

\subsection{Выполнение задания}
\lipsum[1] %TODO

\section{Выводы}
\lipsum[1-4] % TODO chktex 8

\end{document}
