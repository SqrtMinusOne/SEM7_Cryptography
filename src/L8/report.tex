\documentclass[a4paper, 14pt]{extarticle}

\usepackage{../generalPreamble}
\usepackage{../reportFormat}
\usepackage{../sourceCode}

\begin{document}
\begin{titlepage}
    \centering
    {\bfseries
        \uppercase{
            Минобрнауки России \\
            Санкт-Петербургский государственный \\
            Электротехнический университет \\
            \enquote{ЛЭТИ} им. В.И.Ульянова (Ленина)\\
        }
        Кафедра МО ЭВМ

        \vspace{\fill}
        \uppercase{Отчёт} \\
        по лабораторной работе №8 \\
        по дисциплине \enquote{Криптография и защита информации} \\
        Тема: Изучение цифровой подписи
    }

    \vspace{\fill}
    \begin{tabularx}{0.8\textwidth}{l X c r}
        Студент гр. 6304 & & \underline{\hspace{3cm}} & Корытов П.В.\\
        Преподаватель & & \underline{\hspace{3cm}} & Племянников А.К.
    \end{tabularx}

    \vspace{1cm}
    Санкт-Петербург \\
    \the\year{}
\end{titlepage}
\section*{Цель работы}
Исследовать алгоритмы создания и проверки цифровой подписи, алгоритмы генерации ключевых пар RSA, DSA, ECDSA и получить практические навыки работы с ними, в том числе и в программном продукте CrypTool 1.

\section{Генераторы ключевых пар}
\subsection{Основные теоретические положения}
\lipsum[1] %TODO

\subsection{Формулировка задания}
\begin{enumerate}
    \item Перейти к утилите «Digital Signatures/PKI->PKI/Generate…».
    \item Сгенерировать ключевые пары по алгоритмам RSA-2048, DSA-2048, EC-239. Зафиксируйте время генерации в таблице.
    \item С помощью утилиты «Digital Signatures/PKI-> PKI/Display…» вывести сгенерированный открытый ключ и сохранить соответствующий скриншот.
\end{enumerate}

\subsection{Выполнение задания}
\lipsum[1] %TODO

\section{Процессы создания и проверки цифровой подписи}
\subsection{Основые теоретические положения}
\lipsum[1] %TODO

\subsection{Формулировка задания}
\begin{enumerate}
    \item Открыть текст не менее 5000 знаков. Перейти к приложению Digital Signatures/PKI-> Sign Document…
    \item Задайте хэш-функцию, и другие параметры цифровой подписи.
    \item Создайте подпись ключами, сгенерированными в предыдущем задании. Зафиксируйте время создания цифровой подписи для каждого ключа.
    \item Сохраните скриншот цифровой подписи с помощью приложения Digital Signatures/PKI-> Extract Signature.
    \item Выполните процедуру проверки подписи Digital Signatures/PKI-> Verify Signature для случаев сохранения и нарушения целостности исходного текста. Сохраните скриншоты результатов.
\end{enumerate}

\subsection{Ход работы}
\lipsum[1] %TODO

\section{Схемы цифровой подписи на эллиптических кривых}
\subsection{Основые теоретические положения}
\lipsum[1] %TODO

\subsection{Формулировка задания}
\begin{enumerate}
    \item Выполните процедуру создание подписи «Digital Signatures/PKI-> Sign Document…» алгоритмом ECSP-DSA в пошаговом режиме (Display inter.results=ON). Зафиксируйте скриншоты последовательности шагов.
    \item Выполните процедуру проверки подписи ECSP-DSA для случаев сохранения и нарушения целостности исходного текста. Сохраните скриншоты результатов.
    \item Проверить лекционный материал по ECDSA, выполнив создание и проверку подписи сообщения M (принять $M=h(M)$) приложением Indiv.Procedures->Number Theory->Point Addition on EC.\@
\end{enumerate}

\subsection{Ход работы}
\lipsum[1] %TODO

\section{Демострация процесса подписи в среде PKI}
\subsection{Основые теоретические положения}
\lipsum[1] %TODO

\subsection{Формулировка задания}
\begin{enumerate}
    \item Запустить демонстрационную утилиту «Digital Signatures/PKI-> Signature Demonstration…».
    \item Получите сертификат на ранее сгенерированную ключевую пару RSA-2048.
    \item Выполните и сохраните скриншоты всех этапов создания цифровой подписи документа.
    \item Сохраните скриншот сертификата для проверки этой цифровой подписи
\end{enumerate}

\subsection{Ход работы}
\lipsum[1] %TODO

\section{Подписание своего отчёта}
\subsection{Формулировка задания}
\begin{enumerate}
    \item Сконвертируйте отчёт в формат pdf
    \item Экспортируйте ранее созданный сертификат ключевой пары RSA Digital Signatures/PKI->PKI/Generate…->Export PSE (\#PKCS12).
    \item Откройте pdf-версию отчета и попытайтесь подписать с использованием этого сертификата.
    \item Создайте собственный самоподписанный сертификат в среде Adobe Reader и используйте его для подписи отчета.
    \item Сохраните скриншоты свойст подписи и сертификата
    \item Внесите изменения (маркеры, комментарии) в отчёт и проверьте подпись.
\end{enumerate}

\subsection{Ход работы}
\lipsum[1] %TODO

\clearpage
\section{Выводы}
\lipsum[1] %TODO

\end{document}
